\documentclass[final, noposter]{polytech/polytech}

\schooldepartment{di}
\typereport{pldi5}
\reportyear{2018-2019}

\title{Création d'un calendrier de championat}
%\subtitle{}

\student{Thomas}{Couchoud}{thomas.couchoud@etu.univ-tours.fr}
\student{Victor}{Coleau}{victor.coleau@etu.univ-tours.fr}
\academicsupervisor{Christophe}{Lenté}{christophe.lente@univ-tours.fr}

%%%%%%%%%%%%%%%%%%%%%%%%%%%%%%%%%%%%%%%%%%%%%%%%

%TODO
\resume{}
\motcle{}

\abstract{}
\keyword{}

%%%%%%%%%%%%%%%%%%%%%%%%%%%%%%%%%%%%%%%%%%%%%%%%

\begin{document}
\chapter{Introduction}
	\section{Contexte}
	\section{Objectifs}
	\section{Hypothèses}

\chapter{Description générale}	
	\section{Environnement du projet}
	\section{Caractéristiques des utilisateurs}
		Les utilisateurs finaux du programme sont des gens n'ayant aucuun lien avec l'informatique. De ce fait, le programme doit être utilisable de la manière la plus simple et intuitive possible. Pour cela il faut que l'utilisateur est le moins de paramètres à définir en amont à l'exception des fichiers d'entrée à fournir. De plus, il est nécessaire que l'interface graphique soit claire, propre et intuitive ainsi que simple de prise en main et d'utilisation. \\
 		
		Dans le cas où le développement serait par la suite continuer par une autre équipe, le projet fut conçu de sorte à être facilement reprit. Il suffit d'avoir un environnement de développement adapté (Voir \autoref{maven}). De plus, la Javadoc est présente au sein du code afin de facilité sa prise en main.

	\section{Fonctionnalités du système}
		\subsection{Importer les données}

		\subsection{Afficher le planning}

		\subsection{Plannification des matches}

		\subsection{Exporter les résultats}

	\section{Description des interface}
	
\chapter{Réalisation du logiciel}

\chapter{Gestion de projet}

\chapter{Améliorations à prévoir}

\chapter{Conclusion}

\end{document}